
\newcommand{\rarrow}{$\rightarrow$}
\begin{frame}{Progress \& Status}
  \centering
  \footnotesize
  \begin{itemize}
    \item Acquired MC samples of signal and some relevant backgrounds
    \item Basic framework written to easily make distributions of N-Tuple variables as well as values derived from them. \\
      NTuple \rarrow{} (\texttt{filval} C++ Code) \rarrow{} \\
      Root File with Histograms \rarrow{} (Python plotting routines) \rarrow{} \\
      Plots in Jupyter Notebook
  \end{itemize}

  \tiny
  \begin{tabular}{ll}
  Sample Name                                                                                                    & X-section     \\
  /TTTT\_TuneCUETP8M1\_13TeV-amcatnlo-pythia8/RunIISummer16MiniAODv2-PUMoriond17\_80X\                           & 0.009103\pb{} \\
  \_mcRun2\_asymptotic\_2016\_TrancheIV\_v6-v1                                                                    &               \\
  /TTWJetsToLNu\_TuneCUETP8M1\_13TeV-amcatnloFXFX-madspin-pythia8/RunIISummer16MiniAODv2-PUMoriond17\_80X\       & 0.2043\pb{}   \\
  \_mcRun2\_asymptotic\_2016\_TrancheIV\_v6\_ext1-v3                                                              &               \\
  /TTZToLLNuNu\_M-10\_TuneCUETP8M1\_13TeV-amcatnlo-pythia8/RunIISummer16MiniAODv2-PUMoriond17\_80X\              & 0.2529\pb{}   \\
  \_mcRun2\_asymptotic\_2016\_TrancheIV\_v6\_ext1-v1                                                              &               \\
  /ttHJetToNonbb\_M125\_13TeV\_amcatnloFXFX\_madspin\_pythia8\_mWCutfix/RunIISummer16MiniAODv2-PUMoriond17\_80X\ & 0.215\pb{}    \\
  \_mcRun2\_asymptotic\_2016\_TrancheIV\_v6\_ext1-v1                                                              &               \\
  \end{tabular}
\end{frame}
